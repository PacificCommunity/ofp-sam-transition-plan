\documentclass{SCreport}
\usepackage[utf8]{inputenc}
\usepackage[T1]{fontenc}
\setcounter{tocdepth}{2}
\reporttype{SC20}
\reportauthor{A.~Magnusson\footnote{\spc} and N.~Davies\footnote{TeTakina Ltd}}
\reporttitle{Scoping the Next Generation of Tuna Stock Assessment
  Software:\\Progress Report and Outline of Options (Project 123)}
\reportnumber{\tpnum/SA-WP-01}
\reportdate{14 August 2024}
\begin{document}

\wcpfctitlepage

\tableofcontents
\newpage

\section{Executive summary}

\section{Introduction}

\subsection{The need to migrate to new software}

Following the retirement of the lead developer of MULTIFAN-CL (MFCL), Dave
Fournier, future advances to the MFCL software are not expected to be as
mathematically innovative as they were in the past. While this does not render
MFCL obsolete in the medium-term, it flags the need to plan and identify whether
alternative existing software exists, or new software must be developed in the
longer-term, to continue to support the specificities and future requirements of
WCPFC tuna stock assessments.

While MFCL (Fournier et al. 1998) continues to be improved to service the WCPFC
tuna assessment needs over at least the next 5+ years, it is important to start
on a phased approach to its replacement. An initial scoping phase is required to
assess what features and capabilities will be important in future assessment
software for tunas. This scoping phase will benefit from input from stock
assessment scientists across global tuna RFMOs. Once this scoping phase is
conducted, consideration of available software packages in relation to the
desired features and capabilities can be conducted. This may identify suitable
existing software that has potential to provide the desired features and/or has
potential to be developed further. Alternatively, it may indicate whether
embarking on development of a new software package is recommended.

There has also been discussion around the need to explore, through
modeling/simulation exercises, the benefits of applying alternative assessment
structures (i.e., length-age structured versus the traditional length-based
age-structured approach of MFCL and Stock Synthesis) before embarking on major
software developments or changing methodology. Similar can be said about
exploring benefits of state-space models and their use of random variables.
Simulation exercises to explore the benefits or drawbacks of alternative model
structures or approaches will also require collaboration across tuna RFMOs and
experienced practitioners using the alternative approaches and/or software.

An important outcome of this work would be to ultimately have a software package
that has the desired functionality for tuna assessments, not only for WCPFC, but
globally, thus creating a user community and ongoing development support
capacity, so as to avoid the current situation we are facing with MFCL. Wider
collaboration in this venture is essential to achieving this and is expected to
be encouraged through this project.

\subsection{Project outline}

The project is divided into stages, as follows:

Year 2024

\begin{enumerate}
  \item Review and identify a list of necessary and desired features for
  software to conduct tuna stock assessments.
  \item Identify existing software platforms that have some or all these
  features or can be extended to add such features.
  \item Reach out to and initiate collaboration with developers who are already
  exploring extensions to current age-structured catch-at-age and state-space
  models that would allow those models to fit to length data.
  \item Conduct two workshops with selected experts from other tuna RFMOs and or
  with relevant expertise. The first workshop can be remote (prior to SC20) and
  the second one potentially in person (post SC20). The main goal will be to
  plan and document collaborative simulation studies to evaluate the performance
  of candidate software platforms for tuna stock assessments.
\end{enumerate}

Years 2025 and 2026 (subject to SC advice and funding approvals by WCPFC)

\begin{enumerate}\setcounter{enumi}{4}
  \item Conduct the simulation studies outlined in stage 4, in collaboration
  with experts that are supported by other agencies/RFMOs.
  \item Based on the results from simulation studies, determine which, if any,
  software platforms can be considered viable candidate platforms for WCPFC tuna
  assessments.
  \item If a viable candidate platform has been identified, provide a plan to
  transition assessments to this software. Evaluate which transitioned
  assessments could replace old MFCL assessments and build the transition into
  the WCPFC assessment schedule.
  \item If no viable candidate platform is identified, further extensions to
  existing platforms may be required, or the development of a new assessment
  software platform. The design process for a new platform would require
  additional development workshops and involvement of software developers
  (additional funds likely required) and could include identification of MFCL
  features and algorithms that can be borrowed and ported across to a new
  framework.
\end{enumerate}

\subsection{Existing software}

Punt et al. (2020)

\subsection{Recent and ongoing software development}

\section{Possible tasks for SPC to prioritize}

\subsection{Migrating assessments to existing software}

\subsection{Developing new software}

\section{Timeline}

\subsection{International expert meeting 2024}

\subsubsection{Objectives}

The meeting objectives were:

\begin{enumerate}
  \item Communicate SPC scoping project and upcoming explorations, decisions,
  and development.
  \item Discuss succession plans for MULTIFAN-CL as well as Stock Synthesis.
  \item Seek advice from the scientific community.
  \item Seek collaboration with tuna RFMOs and various research labs.
\end{enumerate}

\subsubsection{Format}

Two online meetings were held on 13 May and 18 June, inviting stock assessment
and software development experts from around the world. The two sessions had the
same format and agenda, but one was centered on European time zones and the
other on North American time zones. Around 40 participants represented the tuna
RFMOs (CCSBT, IATTC, ICCAT, IOTC, WCPFC), stock assessment software projects
(ALSCL, CASAL, FIMS, Gadget, MFCL, SAM, sbt, Stock Synthesis, WHAM), and
maintainers of relevant programming languages (ADMB, RTMB, TMB).

The meeting agenda covered the following discussion topics:

\begin{itemize}
  \item Platforms currently used in tuna stock assessments\\[-4ex]
  \item Common challenges for all tuna RFMOs, longevity of Stock Synthesis and
  MULTIFAN-CL, succession plans\\[-4ex]
  \item SPC challenges and project plan\\[-4ex]
  \item Features of current and future platforms\\[-4ex]
  \item Discussion on platform features most relevant for tuna\\[-4ex]
  \item State-space models and latest developments\\[-4ex]
  \item What do you think is the best way forward for SPC?\\[-4ex]
  \item Summary of discussions, next steps, collaboration
\end{itemize}

\subsubsection{Outcomes}

There was a consensus among the experts that the goal should be to design and
develop a model specific for tuna assessments, rather than a general model for
global usage and all species. The advice was to start with a lean design and get
a simple model up and running before adding all the features required for an
assessment. In general, the cost of adding features is much greater than the
implementation cost, as each layer complexity makes long-term maintenance and
future modifications of the software more difficult and costly.

RTMB (Kristensen 2024) is a new alternative interface for developing TMB
(Kristensen et al. 2017) models. RTMB provides a leaner development paradigm
than TMB. The recommendation from the TMB/RTMB development team, articulated at
the expert meeting, was to develop the next tuna stock assessment model in RTMB
rather than TMB.

State-space models are a statistically and computationally efficient way to
allow time-varying processes in stock assessment models, such as time-varying
selectivity. Other statistical approaches exist, but the successful track record
of using state-space models in production assessments, e.g., across a variety of
Atlantic groundfish stocks, indicates that state-space formulation can be
recommended for new model development.

Age-length structure is explicitly tracked in Gadget models, accounting for the
fact that fast-growing individuals are caught by the fishery and the
slow-growing individuals in the cohort remain in the population. Casal and Stock
Synthesis have optional model features with a similar aim. This leads to an
improved level of realism, and simulations with a very basic model ALSCL (Zhang
and Cadigan 2022) indicate improved estimation accuracy. The drawback is that
tracking the population structure in terms of age and length, instead of age
only, comes at a considerable computational cost. The SPC tuna assessment models
that are currently run in MFCL are already very computationally heavy, with
models running overnight before results are available. The recommendation is to
explore the feature of incorporating full age-length structure in the next tuna
assessment models, considering estimation performance and computational time, as
well as software development and maintenance cost.

\subsection{Workshop activities in 2024--2026}

\subsection{Launching the main project}

\section{Required resources}

\subsection{Collaboration with other tRFMOs}

\subsection{SPC staff positions, consultants}

\section{References}

\sloppy\setlength\hyphenpenalty{1000}

\begin{description}\setlength\itemsep{0ex}
  \item Fournier, D.A., J. Hampton, and J.R. Sibert. 1998. MULTIFAN-CL: A
  length-based, age-structured model for fisheries stock assessment, with
  application to South Pacific albacore, \textit{Thunnus alalunga}. Can. J.
  Fish. Aquat. Sci. 55:2105--2116.
  \item Kristensen, K. 2024. RTMB: R bindings for TMB. R package version 1.5.\\
  https://cran.r-project.org/package=RTMB.
  \item Kristensen, K. A. Nielsen, C.W. Berg, H. Skaug, and B.M. Bell. 2016.
  TMB: Automatic differentiation and Laplace approximation. J. Stat. Softw.
  70(5):1--21.
  \item Punt, A.E., A. Dunn, B.Þ. Elvarsson, J. Hampton, S.D. Hoyle, M.N.
  Maunder, R.D. Methot, and A. Nielsen. 2020. Essential features of the
  next-generation integrated fisheries stock assessment package: A perspective.
  Fish. Res. 229:105617.
  \item Zhang, F. and N.G. Cadigan. 2022. An age-and length-structured
  statistical catch-at-length model for hard-to-age fisheries stocks. Fish and
  Fish. 23:1121--1135.
\end{description}

\end{document}
