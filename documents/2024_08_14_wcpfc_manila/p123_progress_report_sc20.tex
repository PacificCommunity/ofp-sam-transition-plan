\documentclass{SCreport}
\setcounter{tocdepth}{2}
\reporttype{SC20}
\reportauthor{A.~Magnusson\footnote{\spc} and N.~Davies\footnote{TeTakina Ltd}}
\reporttitle{Scoping the Next Generation of Tuna Stock Assessment
  Software:\\Progress Report and Outline of Options (Project 123)}
\reportnumber{\tpnum/SA-WP-01}
\reportdate{14 August 2024}
\begin{document}

\wcpfctitlepage

\tableofcontents
\newpage

\section{Executive summary}

\section{Introduction}

\subsection{The need to migrate to new software}

Following the retirement of the lead developer of MULTIFAN-CL (MFCL), Dave
Fournier, future advances to the MFCL software are not expected to be as
mathematically innovative as they were in the past. While this does not render
MFCL obsolete in the medium-term, it flags the need to plan and identify whether
alternative existing software exists, or new software must be developed in the
longer-term, to continue to support the specificities and future requirements of
WCPFC tuna stock assessments.

While MFCL continues to be improved to service the WCPFC tuna assessment needs
over at least the next 5+ years, it is important to start on a phased approach
to its replacement. An initial scoping phase is required to assess what features
and capabilities will be important in future assessment software for tunas. This
scoping phase will benefit from input from stock assessment scientists across
global tuna RFMOs. Once this scoping phase is conducted, consideration of
available software packages in relation to the desired features and capabilities
can be conducted. This may identify suitable existing software that has
potential to provide the desired features and/or has potential to be developed
further. Alternatively, it may indicate whether embarking on development of a
new software package is recommended.

There has also been discussion around the need to explore, through
modeling/simulation exercises, the benefits of applying alternative assessment
structures (i.e., length-age structured versus the traditional length-based
age-structured approach of MFCL and Stock Synthesis) before embarking on major
software developments or changing methodology. Similar can be said about
exploring benefits of state-space models and their use of random variables.
Simulation exercises to explore the benefits or drawbacks of alternative model
structures or approaches will also require collaboration across tuna RFMOs and
experienced practitioners using the alternative approaches and/or software.

An important outcome of this work would be to ultimately have a software package
that has the desired functionality for tuna assessments, not only for WCPFC, but
globally, thus creating a user community and ongoing development support
capacity, so as to avoid the current situation we are facing with MFCL. Wider
collaboration in this venture is essential to achieving this and is expected to
be encouraged through this project.

\subsection{Project outline}

\subsection{Existing software}

\subsection{Recent and ongoing software development}

\section{Possible tasks for SPC to prioritize}

\subsection{Migrating assessments to existing software}

\subsection{Developing new software}

\section{Timeline}

\subsection{International expert meeting 2024}

\subsubsection{Format}

\subsubsection{Outcomes}

\subsection{Workshop activities in 2024--2026}

\subsection{Launching the main project}

\section{Required resources}

\subsection{Collaboration with other tRFMOs}

\subsection{SPC staff positions, consultants}

\section{References}

\sloppy\setlength\hyphenpenalty{1000}

\begin{description}\setlength\itemsep{0ex}
  \item Fournier, D.A., J. Hampton, and J.R. Sibert. 1998. MULTIFAN-CL: a
  length-based, age-structured model for fisheries stock assessment, with
  application to South Pacific albacore, \textit{Thunnus alalunga}. Can. J.
  Fish. Aquat. Sci. 55:2105--2116.
\end{description}

\end{document}
