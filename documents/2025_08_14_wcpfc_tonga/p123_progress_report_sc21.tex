\documentclass{SCreport}
\usepackage[utf8]{inputenc}
\usepackage[T1]{fontenc}
\usepackage{caption}
\usepackage{float}
\captionsetup{font=small}
\setcounter{tocdepth}{2}
\newcommand\github
{https://github.com/PacificCommunity/ofp-sam-transition-plan/blob/main}
\newcommand\h[1]{\hspace{#1}}
\newcommand\present{\github/presentations}
\newcommand\I[1]{\rule{0pt}{#1}}
\reporttype{SC21}
\reportauthor{A.~Magnusson\footnote{\spc}, N.~Davies\footnote{TeTakina Ltd},
  G.~Pilling$^1$, P.~Hamer$^1$}
\reporttitle{Scoping the Next Generation of Tuna Stock Assessment
  Software:\\Progress Report (Project 123)}
\reportnumber{\tpnum/SA-WP-01}
\reportdate{17 July 2025}
\setlength\tolerance{2000}
\hyphenation{}
\begin{document}

\wcpfctitlepage

\tableofcontents
\newpage

\section{Executive summary}

The 3-year Project 123 aims to evaluate features and capabilities that will be
important in future tuna assessments, explore fitting models to tuna data using
existing software platforms, guide decisions on the type of new software
development required, and establish collaboration with tuna Regional Fisheries
Management Organizations (RFMOs) and research labs to achieve these goals.

At SC20, ISG-09 reviewed the scoping of next-generation tuna stock assessment
software and supported prioritizing practical tasks, including transitioning
swordfish and striped marlin assessments to Stock Synthesis and testing
simplified models for yellowfin tuna. Members acknowledged the need to focus on
immediate assessment priorities while keeping longer-term software development
under consideration, depending on available resources and capacity. (SC20
Summary Report, Attachment E).

SC21 will review the progress of Project 123, which explores the transition to
next-generation stock assessment software for tuna fisheries. The project report
i) evaluates the benefits, limitations, uncertainties, and resource implications
associated with each software platform under consideration; ii) evaluates the
feasibility of analyzing tagging data independently from the main stock
assessment models, a potential strategy to reduce model complexity while
maintaining scientific robustness; and iii) identifies key analytical features
and technical capabilities that future stock assessment platforms should
incorporate, such as support for spatial structure, tagging integration, and
flexibility for multi-species and multi-fleet assessments, to ensure that WCPFC
assessments remain scientifically credible, transparent, and adaptable to
evolving fishery and management needs.

SC21 will provide feedback on the progress of the project as needed.

\vspace{4ex}

We invite SC21 to:

\begin{itemize}
  \item note that over the next 5+ years, MULTIFAN-CL will begin to be phased
  out as a software platform for WCPFC tuna and billfish stock assessments;
  \item note that in 2025, the two billfish stock assessments, swordfish and
  striped marlin, transitioned from MULTIFAN-CL to Stock Synthesis;
  \item review and comment on two suggested software development work streams,
  described in this report, providing feedback that will guide the preparation
  of project proposals to be presented to SC22; and
  \item convene an ISWG to discuss the progress and develop activities and a TOR
  for the third year of Project 123.
\end{itemize}

\section{Background}

\section{Progress against TOR items}

\subsection{2024 (second half)}

\subsection{2025 (first half)}

\section{Planned activities}

\subsection{2025 (second half)}

\subsection{2026 (first half)}

\section{Recommendation of project activities}

\subsection{2026 terms of reference}

\section{Updated project diagram}

\section{Note to SC}

\end{document}
