\documentclass[aspectratio=169,fleqn]{beamer}
\usepackage{spc}
\usepackage{graphicx}
\begin{document}

\setbeamertemplate{title page}{
  \usebackgroundtemplate{}
  \begin{picture}(0,0)
    \put(-27.97,-147.11){\includegraphics[height=\paperheight]{front_banner}}
    \put(160,73){\includegraphics[width=24mm]{wcpfc}}
    \put(82,2){
      \begin{minipage}{\textwidth}
        \centering\darkgreen\Large\bf\inserttitle
      \end{minipage}
    }
    \put(82,-68){
      \begin{minipage}{\textwidth}
        \centering\darkgray\fns\bf\insertauthor
      \end{minipage}
    }
    \put(82,\datepos){
      \begin{minipage}{\textwidth}
        \centering\darkblue\tiny\insertdate
      \end{minipage}
    }
  \end{picture}
}

% ______________________________________________________________________________

\begin{frame}
  \title{Scoping the Next Generation of\\
    Tuna Stock Assessment Software\\[1.2ex]
    \it\fns Progress Report (Project 123)}
  \author{Arni Magnusson, Nick Davies,\\[0.4ex]
    Graham Pilling, Paul Hamer}
  \date{SCIENTIFIC COMMITTEE\\[0.4ex]
    TWENTY-FIRST REGULAR SESSION\\[0.4ex]
    Nuku'alofa, Tonga\\[0.4ex]
    13--21 August 2025}
  \titlepage
  \begin{picture}(0,0)
    \put(8,-136){\white\tiny\bf WCPFC-SC21-2025/SA-WP-01}
  \end{picture}
\end{frame}

% ______________________________________________________________________________

\begin{frame}{Overview}
  \begin{itemize}
    \item[] {\bf\darkblue Background}\\[0.1ex]
    \comment{the need to migrate to new software}\\[4ex]
    \item[] {\bf\darkblue Progress against TOR items}\\[0.1ex]
    \comment{evaluation of necessary software features, existing software,
      RTMB}\\[0.1ex]
    \comment{likely next-gen software: IATTC designed tuna model}\\[0.1ex]
    \comment{likely next-gen software: DTU spatio-temporal model (focusing on
      tags and movement)}\\[0.1ex]
    \comment{less likely: FIMS tuna-specific modules}\\[4ex]
    \item[] {\bf\darkblue Planned activities 2025--2026}\\[0.1ex]
    \comment{collaborate with IATTC and DTU on designing and developing
      models}\\[0.1ex]
    \comment{funding required next year, software development project proposal
      at SC22}\\[2ex]
  \end{itemize}
\end{frame}

% ______________________________________________________________________________

\begin{frame}{The need to to migrate to new software}\small
  \begin{itemize}
    \item[] MULTIFAN-CL (MFCL) has been used in SPC tuna assessments since
    1990s\\[5ex]
    \item[] MFCL team (Dave Fournier, John Hampton, Nick Davies) retiring in the
    2020s\\[5ex]
    \item[] Development of new features is slowing down\\[5ex]
    \item[] Resources are being allocated to succession plans\\[4ex]
  \end{itemize}
\end{frame}

% ______________________________________________________________________________

\begin{frame}{Migrating all MFCL assessments to other platforms}\small
  \textbf{\darkgreen Shared} process, {\darkgreen\bf continuous} communications,
  {\darkgreen\bf adaptive} strategy\\[3ex]
  \begin{itemize}
    \item[] {\blue\bf WCPFC} ~--~ guidance\\[2ex]
    \item[] {\blue\bf SPC} ~--~ conduct and coordinate the work\\[4ex]
  \end{itemize}
  Also involved: other tuna RFMOs and various research labs\\[4ex]
  Possibly different software platforms for different stocks\\[4ex]
\end{frame}

% ______________________________________________________________________________

\begin{frame}{Project outline}\small
  This scoping project is scheduled from 1 Feb 2024 to 31 Dec 2026. It
  will:\\[3ex]
  \begin{itemize}
    \item[] Evaluate features and capabilities that will be important in future
    tuna assessments\\[3ex]
    \item[] Explore fitting models to tuna data using existing software
    platforms\\[3ex]
    \item[] Guide decisions on what kind of new software development will be
    required\\[3ex]
    \item[] Establish collaboration with tuna RFMOs and research labs to achieve
    these goals\\[3ex]
  \end{itemize}
\end{frame}

% ______________________________________________________________________________

\begin{frame}{TOR summary}\small
  \textit{\darkgray\fns Work areas from Aug 2024 to Aug 2025:}\\[1ex]
  \begin{enumerate}
    \item Evaluate {\darkgreen necessary features} for tuna assessment software
    and {\darkgreen existing software}\\[1.5ex]
    \item Conduct 2024 {\darkgreen workshops}\\[1.5ex]
    \item Evaluate {\darkgreen RTMB} programming environment\\[1.5ex]
    \item Explore feasibility of enhancing {\darkgreen tagging module} in Stock
    Synthesis\\[1.5ex]
    \item Support {\darkgreen billfish assessments} transitioning from MFCL to
    Stock Synthesis\\[1.5ex]
    \item Prepare a {\darkgreen single-region YFT} dataset for comparing and
    testing platforms\\[1.5ex]
    \item Prepare work plans and budget scenarios for {\darkgreen upcoming
      development project}\\[1.5ex]
    \item Communicate with tuna RFMOs and research labs to establish {\darkgreen
      who will participate}\\[1.5ex]
    \item Communicate with tuna RFMOs to evaluate {\darkgreen FIMS development
      option}\\[3ex]
  \end{enumerate}
\end{frame}

% ______________________________________________________________________________

\begin{frame}{\gray TOR summary}\small
  \textit{\gray\fns Work areas from Aug 2024 to Aug 2025:}\\[1ex]
  \begin{enumerate}
    \item[\orange $\Rightarrow$ 1.] Evaluate {\orange necessary features} for
    tuna assessment software and {\orange existing software}\\[1.5ex]
    \item[\gray 2.] \gray Conduct 2024 workshops\\[1.5ex]
    \item[\gray 3.] Evaluate RTMB programming environment\\[1.5ex]
    \item[\gray 4.] \gray Explore feasibility of enhancing tagging module in
    Stock Synthesis\\[1.5ex]
    \item[\gray 5.] \gray Support billfish assessments transitioning from MFCL
    to Stock Synthesis\\[1.5ex]
    \item[\gray 6.] Prepare a single-region YFT dataset for comparing and
    testing platforms\\[1.5ex]
    \item[\orange $\Rightarrow$ 7.] \black Prepare work plans and budget
    scenarios for {\orange upcoming development project}\\[1.5ex]
    \item[\gray 8.] \gray Communicate with tuna RFMOs and research labs to
    establish who will participate\\[1.5ex]
    \item[\gray 9.] Communicate with tuna RFMOs to evaluate FIMS development
    option\\[3ex]
  \end{enumerate}
\end{frame}

% ______________________________________________________________________________

\begin{frame}{Necessary software features for tuna assessments}\small
  A long wish list of next-generation software features was covered in the CAPAM
  2019 review paper (Punt et al. 2020)\\[2.5ex]
  This year's P123 progress report to SC21 identifies features that are
  especially relevant for\\
  tuna assessments:\\[1.5ex]
  \begin{itemize}
    \item Incorporate length comps, weight comps, tags, otoliths, close-kin
    mark-recapture data (!)\\[2ex]
    \item Age-specific M, length-specific selectivity, sex-specific growth and
    M, region-specific growth\\[2ex]
    \item Explicit regions with movement, time steps within a year, age and
    length structure (?)\\[2ex]
    \item Multispecies interactions (?), climate change (?)\\[2ex]
    \item Random effects (=\,state space), parallel computing, computation
    time\\[2ex]
  \end{itemize}
\end{frame}

% ______________________________________________________________________________

\begin{frame}{Necessary software features for tuna assessments}\small
  \textdarkgreen{Three general criteria}\\[1ex]
  \begin{enumerate}
    \item \textit{Scientific quality}:
    The software should have good statistical performance,\\
    make good use of available data, allow spatial and temporal variability,\\
    be computationally efficient, and provide scientific advice that matches\\
    the needs for the management of each stock.\\[2.5ex]
    \item \textit{Beginner friendly}:
    The user interface and auxiliary tools should allow\\
    new scientific staff to have a good understanding of the configuration\\
    of model options and be able to work efficiently with the model.\\[2.5ex]
    \item \textit{Widely used}:
    SPC does not want to use stock assessment software in\\
    isolation, the software should have a wider development team and user\\
    community beyond SPC, new scientific staff can find expert help and\\
    technical advice outside of SPC.\\[2.5ex]
  \end{enumerate}
\end{frame}

% ______________________________________________________________________________

\begin{frame}{Project website}
  \centering\small
  \textblue{\url{https://github.com/PacificCommunity/ofp-sam-transition-plan}}
\end{frame}

\end{document}
