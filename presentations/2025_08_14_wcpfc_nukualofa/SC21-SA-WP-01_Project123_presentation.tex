\documentclass[aspectratio=169,fleqn]{beamer}
\usepackage{spc}
\usepackage{graphicx}
\begin{document}

\setbeamertemplate{title page}{
  \usebackgroundtemplate{}
  \begin{picture}(0,0)
    \put(-27.97,-147.11){\includegraphics[height=\paperheight]{front_banner}}
    \put(160,73){\includegraphics[width=24mm]{wcpfc}}
    \put(82,2){
      \begin{minipage}{\textwidth}
        \centering\darkgreen\Large\bf\inserttitle
      \end{minipage}
    }
    \put(82,-68){
      \begin{minipage}{\textwidth}
        \centering\darkgray\fns\bf\insertauthor
      \end{minipage}
    }
    \put(82,\datepos){
      \begin{minipage}{\textwidth}
        \centering\darkblue\tiny\insertdate
      \end{minipage}
    }
  \end{picture}
}

% ______________________________________________________________________________

\begin{frame}
  \title{Scoping the Next Generation of\\
    Tuna Stock Assessment Software\\[1.2ex]
    \it\fns Progress Report (Project 123)}
  \author{Arni Magnusson, Nick Davies,\\[0.4ex]
    Graham Pilling, Paul Hamer}
  \date{SCIENTIFIC COMMITTEE\\[0.4ex]
    TWENTY-FIRST REGULAR SESSION\\[0.4ex]
    Nuku'alofa, Tonga\\[0.4ex]
    13--21 August 2025}
  \titlepage
  \begin{picture}(0,0)
    \put(8,-136){\white\tiny\bf WCPFC-SC21-2025/SA-WP-01}
  \end{picture}
\end{frame}

% ______________________________________________________________________________

\begin{frame}{Overview}
  \begin{itemize}
    \item[] {\bf\darkblue Background}\\[0.1ex]
    \comment{the need to migrate to new software}\\[4ex]
    \item[] {\bf\darkblue Progress against TOR items}\\[0.1ex]
    \comment{evaluation of necessary software features, existing software,
      RTMB}\\[0.1ex]
    \comment{likely next-gen software: IATTC designed tuna model}\\[0.1ex]
    \comment{likely next-gen software: DTU spatio-temporal model (focusing on
      tags and movement)}\\[0.1ex]
    \comment{less likely: FIMS tuna-specific modules}\\[4ex]
    \item[] {\bf\darkblue Planned activities 2025--2026}\\[0.1ex]
    \comment{collaborate with IATTC and DTU on designing and developing
      models}\\[0.1ex]
    \comment{funding required next year, software development project proposal
      at SC22}\\[2ex]
  \end{itemize}
\end{frame}

% ______________________________________________________________________________

\begin{frame}{The need to to migrate to new software}\small
  \begin{itemize}
    \item[] MULTIFAN-CL (MFCL) has been used in SPC tuna assessments since
    1990s\\[5ex]
    \item[] MFCL team (Dave Fournier, John Hampton, Nick Davies) retiring in the
    2020s\\[5ex]
    \item[] Development of new features is slowing down\\[5ex]
    \item[] Resources are being allocated to succession plans\\[5ex]
    \item[] The overarching objective is to
    {\darkgreen transition all assessments from MFCL}\\[4ex]
  \end{itemize}
\end{frame}

% ______________________________________________________________________________

\begin{frame}{Project outline}\small
  This scoping project is scheduled from 1 Feb 2024 to 31 Dec 2026. It
  will:\\[3ex]
  \begin{itemize}
    \item[] {\darkgreen Evaluate features} and capabilities that will be
    important in future tuna assessments\\[3ex]
    \item[] Explore fitting models to tuna data using {\darkgreen existing
      software platforms}\\[3ex]
    \item[] Guide decisions on what kind of
    {\darkgreen new software development} will be required\\[3ex]
    \item[] {\darkgreen Establish collaboration} with tuna RFMOs and research
    labs to achieve these goals\\[3ex]
  \end{itemize}
\end{frame}

% ______________________________________________________________________________

\begin{frame}{TOR summary}\small
  \textit{\darkgray\fns Work areas from Aug 2024 to Aug 2025:}\\[1ex]
  \begin{enumerate}
    \item Evaluate {\darkgreen necessary features} for tuna assessment software
    and {\darkgreen existing software}\\[1.5ex]
    \item Conduct 2024 {\darkgreen workshops}\\[1.5ex]
    \item Evaluate {\darkgreen RTMB} programming environment\\[1.5ex]
    \item Explore feasibility of enhancing {\darkgreen tagging module} in Stock
    Synthesis\\[1.5ex]
    \item Support {\darkgreen billfish assessments} transitioning from MFCL to
    Stock Synthesis\\[1.5ex]
    \item Prepare a {\darkgreen single-region YFT} dataset for comparing and
    testing platforms\\[1.5ex]
    \item Prepare work plans and budget scenarios for {\darkgreen upcoming
      development project}\\[1.5ex]
    \item Communicate with tuna RFMOs and research labs to establish {\darkgreen
      who will participate}\\[1.5ex]
    \item Communicate with tuna RFMOs to evaluate {\darkgreen FIMS development
      option}\\[3ex]
  \end{enumerate}
\end{frame}

% ______________________________________________________________________________

\begin{frame}{\gray TOR summary}\small
  \textit{\gray\fns Work areas from Aug 2024 to Aug 2025:}\\[1ex]
  \begin{enumerate}
    \item[\orange $\Rightarrow$ 1.] Evaluate {\orange necessary features} for
    tuna assessment software and {\orange existing software}\\[1.5ex]
    \item[\gray 2.] \gray Conduct 2024 workshops\\[1.5ex]
    \item[\gray 3.] Evaluate RTMB programming environment\\[1.5ex]
    \item[\gray 4.] \gray Explore feasibility of enhancing tagging module in
    Stock Synthesis\\[1.5ex]
    \item[\gray 5.] \gray Support billfish assessments transitioning from MFCL
    to Stock Synthesis\\[1.5ex]
    \item[\gray 6.] Prepare a single-region YFT dataset for comparing and
    testing platforms\\[1.5ex]
    \item[\orange $\Rightarrow$ 7.] \black Prepare work plans and budget
    scenarios for {\orange upcoming development project}\\[1.5ex]
    \item[\gray 8.] \gray Communicate with tuna RFMOs and research labs to
    establish who will participate\\[1.5ex]
    \item[\gray 9.] Communicate with tuna RFMOs to evaluate FIMS development
    option\\[3ex]
  \end{enumerate}
\end{frame}

% ______________________________________________________________________________

\begin{frame}{Evaluation of necessary software features}\small
  A long wish list of next-generation software features was covered in the CAPAM
  2019 review paper (Punt et al. 2020)\\[2.5ex]
  This year's P123 progress report to SC21 identifies features that are
  especially relevant for\\
  tuna assessments:\\[1.5ex]
  \begin{itemize}
    \item Incorporate length comps, weight comps, tags, otoliths, close-kin
    mark-recapture data (!)\\[2ex]
    \item Age-specific M, length-specific selectivity, sex-specific growth and
    M, region-specific growth\\[2ex]
    \item Explicit regions with movement, time steps within a year, age and
    length structure (?)\\[2ex]
    \item Multispecies interactions (?), climate change (?)\\[2ex]
    \item Random effects (=\,state space), parallel computing, computation
    time\\[2ex]
  \end{itemize}
\end{frame}

% ______________________________________________________________________________

\begin{frame}{Evaluation of necessary software features}\small
  \textdarkgreen{\bf Three main criteria}\\[1ex]
  \begin{enumerate}
    \item \textit{Scientific quality}:
    The software should have good statistical performance,\\
    make good use of available data, allow spatial and temporal variability,\\
    be computationally efficient, and provide scientific advice that matches\\
    the needs for the management of each stock.\\[2.5ex]
    \item \textit{Beginner friendly}:
    The user interface and auxiliary tools should allow\\
    new scientific staff to have a good understanding of the configuration\\
    of model options and be able to work efficiently with the model.\\[2.5ex]
    \item \textit{Widely used}:
    SPC does not want to use stock assessment software in\\
    isolation, the software should have a wider development team and user\\
    community beyond SPC, especially at tRFMOs, so new scientific staff\\
    can find expert help and technical advice outside of SPC.\\[2.5ex]
  \end{enumerate}
\end{frame}

% ______________________________________________________________________________

\begin{frame}{Evaluation of existing software}\small
  \textdarkgreen{\bf Stock Synthesis}\\[2ex]
  Used in tuna assessments by the IATTC, IOTC, and ICCAT, and in WCPFC billfish
  assessments\\[2.5ex]
  Two options for incorporate tagging data: (1) convert release lengths to age
  externally, or\\
  (2) analyze the tags externally and incorporate as abundance indices.
  Especially relevant for WCPO skipjack assessment\\[2.5ex]
  The SS3 development team does not plan to add new major features such as the
  ability to incorporate CKMR data, which is especially relevant for future
  South Pacific albacore assessment\\[2.5ex]
  The recent use of SS3 at SPC has had a positive impact on collaboration
  between the tRFMOs, relevant for the discussion, development, and
  tests of future software\\[2.5ex]
  Stock Synthesis scores particularly high in two key criteria: the ability of
  new staff scientists to use it effectively and the worldwide user community,
  including tuna RFMOs.\\[2ex]
\end{frame}

% ______________________________________________________________________________

\begin{frame}{Evaluation of existing software}\small
  \textdarkgreen{\bf Gadget}\\[2ex]
  Explicit age-length structured platform\\[2.5ex]
  Gadget 3 was recently implemented in TMB, resulting in substantial performance
  gains\\[2.5ex]
  Wide range of features relevant for tuna assessments and a plan
  is underway to test the use\\
  of Gadget on the single-region yellowfin tuna dataset\\[2.5ex]
  A useful reference for designing any future tuna assessment software\\[2.5ex]
  Gadget scores particularly high for representing state-of-the-art statistical
  methods of interest: explicit age-length structure and a TMB implementation
  that may allow the use of random effects to have processes vary with time
  and/or between regions\\[8ex]
\end{frame}

% ______________________________________________________________________________

\begin{frame}{Evaluation of existing software}\small
  \textdarkgreen{\bf SBT}\\[2ex]
  Used in the assessment of southern bluefin tuna by CCSBT\\[2.5ex]
  Implemented as an R package based on RTMB\\[2.5ex]
  Fitted to CKMR, a new and important data type in future South Pacific albacore
  assessments\\[2.5ex]
  The SBT package is designed for a single-region assessment and would require
  some additional development to be used for a multiregion assessment\\[2.5ex]
  The scoping project has reached out to the SBT team and will discuss further
  the possibility of using their model code as a starting point for developing
  new software for the South Pacific albacore assessment\\[2.5ex]
  SBT scores particularly high for incorporating CKMR data in a tuna stock
  assessment\\[2ex]
\end{frame}

% ______________________________________________________________________________

\begin{frame}{Evaluation of existing software}\small
  \textdarkgreen{\bf FIMS}\\[2ex]
  NOAA project developing a new framework of stock assessment models based on
  TMB\\[2.5ex]
  Initial development and case studies have focused on models fitting to age
  composition data,\\
  but the design includes the ability to fit to length composition data.
  Spatially explicit models\\
  and tagging data are currently not high priorities for the FIMS
  project\\[2.5ex]
  Might be possible to design and develop tuna-specific FIMS modules: basic
  model extensions and/or introducing underlying changes in the model
  structure.\\[2.5ex]
  The FIMS project might develop an overly complex framework architecture that
  results in slower progress and lower levels of code contributions than
  anticipated. A possible outcome could be that FIMS software cannot be used in
  future tuna assessments\\[8ex]
\end{frame}

% ______________________________________________________________________________

\begin{frame}{Evaluation of existing software}\small
  \textdarkgreen{\bf Other}\\[2ex]
  Casal 2 is the latest version of the Casal stock assessment platform,
  rewritten with an\\
  improved design and user interface\\[2.5ex]
  Wide range of features relevant for tuna assessments\\[2.5ex]
  However, its strengths and features fall slightly short of Gadget 3 for the
  purposes of this\\
  scoping project\\[20ex]
\end{frame}

% ______________________________________________________________________________

\begin{frame}{\gray TOR summary}\small
  \textit{\gray\fns Work areas from Aug 2024 to Aug 2025:}\\[1ex]
  \begin{enumerate}
    \item[\orange $\Rightarrow$ 1.] Evaluate {\orange necessary features} for
    tuna assessment software and {\orange existing software}\\[1.5ex]
    \item[\gray 2.] \gray Conduct 2024 workshops\\[1.5ex]
    \item[\gray 3.] Evaluate RTMB programming environment\\[1.5ex]
    \item[\gray 4.] \gray Explore feasibility of enhancing tagging module in
    Stock Synthesis\\[1.5ex]
    \item[\gray 5.] \gray Support billfish assessments transitioning from MFCL
    to Stock Synthesis\\[1.5ex]
    \item[\gray 6.] Prepare a single-region YFT dataset for comparing and
    testing platforms\\[1.5ex]
    \item[\orange $\Rightarrow$ 7.] \black Prepare work plans and budget
    scenarios for {\orange upcoming development project}\\[1.5ex]
    \item[\gray 8.] \gray Communicate with tuna RFMOs and research labs to
    establish who will participate\\[1.5ex]
    \item[\gray 9.] Communicate with tuna RFMOs to evaluate FIMS development
    option\\[3ex]
  \end{enumerate}
\end{frame}

% ______________________________________________________________________________

\begin{frame}{Upcoming development project}\small
  Development project proposal to be \textdarkgreen{\bf submitted to SC22}, as a
  key outcome from\\
  the scoping project\\[2ex]
  As an initial estimate, the development project could be launched as a
  three-year project\\
  from 2027 to 2030, with an annual budget of \$200,000. This would
  allow:\\[0.5ex]
  \begin{itemize}
    \item[] \$50,000 per year for workshops and smaller development subprojects,
    in the same\\
    manner as the current scoping project; and\\[1.5ex]
    \item[] \$150,000 per year to divide between the development options that
    will be prioritized.\\[3ex]
  \end{itemize}
  A preliminary recommendation from the scoping project is to have the
  development project pursue primarily two work streams:\\[0.5ex]
  \begin{itemize}
    \item DTU spatio-temporal model
    \item IATTC designed tuna model\\[1ex]
  \end{itemize}
\end{frame}

% ______________________________________________________________________________

\begin{frame}{Upcoming development project}\small
  \textdarkgreen{\bf DTU spatio-temporal model}\\[2ex]
  New collaboration between SPC and statisticians at the Technical University of
  Denmark (DTU), Tobias Mildenberger and Anders Nielsen\\[2.5ex]
  A workshop was conducted in Copenhagen in May 2025, focusing on the fitting a
  spatio-temporal model for analyzing WCPO skipjack tagging data\\[2.5ex]
  Following a similar DTU spatio-temporal model that has been used to analyze
  EPO skipjack tags, producing abundance indices that were used in the 2024
  EPO skipjack assessment\\[18ex]
\end{frame}

% ______________________________________________________________________________

\begin{frame}{Upcoming development project}\small
  \textdarkgreen{\bf IATTC designed tuna model}\\[2ex]
  Scoping project has reached out to IATTC regularly to discuss future software
  for tuna assessments\\[2.5ex]
  In the latest round of discussions in July 2025, Mark Maunder proposed an
  initial design of a platform that could be developed for tuna
  assessments:
  \begin{itemize}
    \item[-] New codebase in RTMB that will be relatively small, easy to modify
    and extend
    \item[-] Keeping it simple, just focusing on the priority needs of tuna
    assessments
    \item[-] Random effects, useful for allowing processes to vary in time and
    possibly between regions\\[3ex]
  \end{itemize}
  \textdarkgreen{Online CAPAM workshop} in December 2025 on the model design and
  a development plan\\[2.5ex]
  $\Rightarrow$ {\darkgreen Small Informal Working Group} discussion at
  SC21\\[3ex]
\end{frame}

% ______________________________________________________________________________

\begin{frame}{General plan}\small
  \begin{tabular}{ll}
    1990s\,--\,2024   & {\darkgreen\it The past.} SPC uses MULTIFAN-CL for all
                        tuna and billfish assessments\\[4ex]
    2025\,--\,c2030   & {\darkgreen\it The interim period.} SPC uses a
                        combination of MULTIFAN-CL and\\[0.4ex]
    ~                 & \,Stock Synthesis for tuna and billfish
                        assessments\\[4ex]
    c2030\,--\,onward & {\darkgreen\it The future.} SPC starts replacing some of
                        the MULTIFAN-CL and\\[0.4ex]
    ~                 & \,Stock Synthesis assessments with new software
  \end{tabular}
  ~\\[6ex]
  \h{1.5ex}The year `c2030' is a placeholder; the exact year is not important
\end{frame}

% ______________________________________________________________________________

\begin{frame}{General plan}\scriptsize
  \h{-2ex}\begin{tabular}{|l|l|l|l|l|l|l|}
    \hline
    \darkgray Year\I{2.4ex}
    & \darkblue Skipjack & \darkblue Yellowfin & \darkblue Bigeye
    & \darkblue Albacore & \green Swordfish    & \green Striped marlin\\
    \hline
    2025\I{2.5ex}
    & MFCL &           &         &          & SS3       & SS3           \\
    \hline
    2026\I{2.5ex}
    &      & MFCL      & MFCL    &          &           &               \\
    \hline
    2027\I{2.5ex}
    &      &           &         & MFCL/New$^2$ &       &               \\
    \hline
    2028\I{2.5ex}
    & MFCL/SS3/New$^1$ & &       &          &           &               \\
    \hline
    2029\I{2.5ex}
    &      & MFCL/SS3/New & MFCL/SS3/New &  & SS3/New   & SS3/New       \\
    \hline
    2030\I{2.5ex}
    &      &              &              & MFCL/New$^2$ & &             \\
    \hline
  \end{tabular}
  \fns\\[3ex]
  \h{1.5ex}1\h{0.4pt}: Skipjack tags could be analyzed externally using the DTU
  spatio-temporal model before 2028,\\
  \h{4.1ex}producing abundance indices for the MFCL/SS3/New stock assessment
  platform.\\[2ex]
  \h{1.5ex}2\h{0.4pt}: Albacore CKMR data could be incorporated in a New stock
  assessment platform, most likely\\
  \h{4.1ex}in 2030.
\end{frame}

% ______________________________________________________________________________

\begin{frame}{Project website}
  \centering\small
  \textblue{\url{https://github.com/PacificCommunity/ofp-sam-transition-plan}}
\end{frame}

\end{document}
