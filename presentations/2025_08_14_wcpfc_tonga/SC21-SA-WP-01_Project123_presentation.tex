\documentclass[aspectratio=169,fleqn]{beamer}
\usepackage{spc}
\usepackage{graphicx}
\begin{document}

\begin{frame}
  \title{\vspace{-5ex}\darkblue Scoping the next stock assessment
    platform\\[2ex]
    \it\large\darkgray
    Project 123 progress update}
  \author{\vspace{-10ex}\darkgray\bf
    Arni Magnusson, Nick Davies,\\[0.5ex]
    Graham Pilling, Paul Hamer}
  \date{\darkgreen WCPFC-SC21-2025/SA-WP-01\\[0.5ex]
    Nuku'alofa, 14 August 2025}
  \titlepage
\end{frame}

% ______________________________________________________________________________

\begin{frame}{Overview}
  \begin{itemize}
    \item[] {\bf\darkblue Background}\\[0.1ex]
    \comment{the need to migrate to new software}\\[4ex]
    \item[] {\bf\darkblue Progress against TOR items}\\[0.1ex]
    \comment{evaluation of necessary software features, existing software,
      RTMB}\\[0.1ex]
    \comment{likely next-gen software: IATTC designed tuna model}\\[0.1ex]
    \comment{likely next-gen software: DTU spatio-temporal model (focusing on
      tags and movement)}\\[0.1ex]
    \comment{less likely: FIMS tuna-specific modules}\\[4ex]
    \item[] {\bf\darkblue Planned activities 2025--2026}\\[0.1ex]
    \comment{collaborate with IATTC and DTU on designing and developing
      models}\\[0.1ex]
    \comment{funding required next year, software development project proposal
      at SC22}\\[3ex]
  \end{itemize}
\end{frame}

% ______________________________________________________________________________

% \begin{frame}{The need to to migrate to new software}
%   \begin{itemize}
%     \item[] MULTIFAN-CL (MFCL) has been used in SPC tuna assessments since
%     1990s\\[4ex]
%     \item[] MFCL team (Dave Fournier, John Hampton, Nick Davies) retiring in the
%     2020s\\[4ex]
%     \item[] Development of new features is slowing down\\[4ex]
%     \item[] Resources are being allocated to succession plans\\[2ex]
%   \end{itemize}
% \end{frame}
% 
% % ______________________________________________________________________________
% 
% \begin{frame}{Migrating all MFCL assessments to other platforms}
%   \textbf{\darkgreen Shared} process, {\darkgreen\bf continuous} communications,
%   {\darkgreen\bf adaptive} strategy\\[3ex]
%   \begin{itemize}
%     \item[] {\blue\bf WCPFC} ~--~ guidance\\[2ex]
%     \item[] {\blue\bf SPC} ~--~ conduct and coordinate the work\\[4ex]
%   \end{itemize}
%   Also involved: other tuna RFMOs and various research labs\\[4ex]
%   Possibly different software platforms for different stocks\\[4ex]
% \end{frame}
% 
% % ______________________________________________________________________________
% 
% \begin{frame}{Project outline}
%   This scoping project is scheduled from 1 Feb 2024 to 31 Dec 2026. It
%   will:\\[3ex]
%   \begin{itemize}
%     \item[] Evaluate features and capabilities that will be important in future
%     tuna assessments\\[3ex]
%     \item[] Explore fitting models to tuna data using existing software
%     platforms\\[3ex]
%     \item[] Guide decisions on what kind of new software development will be
%     required\\[3ex]
%     \item[] Establish collaboration with tuna RFMOs and research labs to achieve
%     these goals\\[3ex]
%   \end{itemize}
% \end{frame}
% 
% % ______________________________________________________________________________
% 
% \begin{frame}{Possible tasks for SPC to prioritize}
%   Subject to SC advice and funding approvals by WCPFC:\\[3ex]
%   \begin{itemize}
%     \item {\orange\it Migrate assessments to existing software}\\[2ex]
%     \item {\orange\it Model exploration using existing software}\\[2ex]
%     \item {\orange\it Extend existing software}\\[2ex]
%     \item {\orange\it Design and develop new software for tuna
%       assessments}\\[2ex]
%   \end{itemize}
% \end{frame}
% 
% % ______________________________________________________________________________
% 
% \begin{frame}{2024 project activities}
%   \begin{itemize}
%     \item[] Scoping project launched (Feb)\\[3ex]
%     \item[] Pre-assessment workshop (Mar)\\[3ex]
%     \item[] International expert meeting (May--Jun)\\[3ex]
%     \item[] SC20 discussion (Aug)\\[3ex]
%     \item[] Developer workshop (Aug--Sep)\\[3ex]
%     \item[] Follow up with tuna RFMOs and research labs (Oct-Dec)\\[2ex]
%   \end{itemize}
% \end{frame}
% 
% % ______________________________________________________________________________
% 
% \begin{frame}{International expert meeting 2024}
%   \begin{itemize}
%     \item[] Invited stock assessment and software development experts from
%     around the world\\
%     \comment{tuna RFMOs and various research labs}\\
%     \comment{stock assessment software projects, relevant programming
%       environments}\\[4ex]
%     \item[] Around 40 participants, two sessions covering European and N
%     American time zones\\[4ex]
%     \item[] Objectives: communicate, discuss, seek advice and
%     collaboration\\[4ex]
%     \item[] Outcomes: {\darkgreen\bf recommendations}, expressed interest in
%     {\darkgreen\bf collaboration} among scientists\\[2ex]
%   \end{itemize}
% \end{frame}
% 
% % ______________________________________________________________________________
% 
% \begin{frame}{Timeline}
%   \begin{itemize}
%     \item[] 2024 \quad Scoping project launched\\[1ex]
%     \item[] \phantom{2024} \quad Pre-assessment workshop\\[1ex]
%     \item[] \phantom{2024} \quad International expert meeting\\[1ex]
%     \item[] \phantom{2024} \quad SC20 discussion\\[3ex]
%     \item[] 2024--2026\quad Workshops\\[3ex]
%     \item[] 2024--2026\quad Launching the {\darkgreen\bf main project}\\[3ex]
%   \end{itemize}
% \end{frame}
% 
% % ______________________________________________________________________________
% 
% \begin{frame}{The main project}
%   The current annual budget of project 123 is sufficient for {\darkgreen\bf
%     scoping the needs},\\[0.5ex]
%   identifying current software platforms, reaching out to the scientific
%   community\\[0.5ex]
%   for consultation, and to conduct occasional workshops to strengthen\\[0.5ex]
%   collaboration ties and initial explorations.\\[5ex]
%   The goal of the {\darkgreen\bf main project}, which could either overlap or
%   succeed the scoping project,\\[0.5ex]
%   is to test/develop tuna stock assessment software and transition all SPC
%   assessments\\[0.5ex]
%   from MFCL to other platforms.\\[2ex]
% \end{frame}
% 
% % ______________________________________________________________________________
% 
% \begin{frame}{Required resources}
%   The overarching objective of transitioning all WCPFC assessments from
%   MULTIFAN-CL\\[0.5ex]
%   to other software platforms will require a larger project with
%   additional project resources\\[0.5ex]
%   beyond the standard service provision agreement for stock assessments, to
%   allow some\\[0.5ex]
%   staff to focus on model exploration and software development.
% \end{frame}
% 
% % ______________________________________________________________________________
% 
% \begin{frame}{Collaboration with other tuna RFMOs}
%   Other tuna RFMOs use primarily Stock Synthesis for tuna assessments,\\[0.5ex]
%   a platform that is also expected to be phased out in the not-too-distant
%   future.\\[0.5ex]
%   Therefore, it would make sense for WCPFC and other tuna RFMOs to
%   coordinate\\[0.5ex]
%   and collaborate in software succession plans and new software
%   development.\\[5ex]
%   Ideally, each tuna RFMO could hire/assign one full-time person to the
%   project\\[0.5ex]
%   for 5 years, or until assessments have been transitioned to the new
%   software.\\[4ex]
% \end{frame}
% 
% % ______________________________________________________________________________
% 
% \begin{frame}{SPC staff positions and consultants}
%   Compared to the other tuna RFMOs, there is greater urgency for WCPFC\\[0.5ex]
%   to move this project forward.\\[4ex]
% 
%   Independent of decisions and commitments of the other tuna RFMOs, the\\[0.5ex]
%   main project would probably require one staff to be dedicated to this
%   work\\[0.5ex]
%   initially and, depending on the direction taken, an additional staff or
%   consultant\\[0.5ex]
%   with software development skills.\\[4ex]
% 
%   It is likely that transitioning MFCL assessments to other software is at
%   least\\[0.5ex]
%   a 5 year proposition.\\[1ex]
% \end{frame}
% 
% % ______________________________________________________________________________
% 
% \begin{frame}{Scientific quality and rate of progress}
%   The resources committed to the main project will determine the\\[0.5ex]
%   scientific quality of the end result and the number of years it takes\\[0.5ex]
%   to transition all SPC assessments from MFCL to other platforms.\\[5ex]
%   Now that the first author of MFCL, David Fournier, has retired, it would
%   be\\[0.5ex]
%   highly beneficial for the project to move relatively fast, before the
%   remaining\\[0.5ex]
%   MFCL team (John Hampton and Nick Davies) will retire and no longer be\\[0.5ex]
%   available for consultation and involvement regarding software design,
%   testing\\[0.5ex]
%   and technical decisions.
% \end{frame}
% 
% % ______________________________________________________________________________
% 
% \begin{frame}{Summary}
%   \begin{itemize}
%     \item[] {\bf\darkblue Introduction} \comment{background, project outline,
%       existing software, new development}\\[5ex]
%     \item[] {\bf\darkblue Possible Tasks} \comment{migrate assessments to
%       existing software, model exploration,\\
%       \h{19.5ex}software development}\\[5ex]
%     \item[] {\bf\darkblue Timeline} \comment{PAW 2024, expert meeting 2024,
%       workshops 2024--2026,\\
%       \h{13ex}launching the main project}\\[5ex]
%     \item[] {\bf\darkblue Required Resources} \comment{collaboration with other
%       tuna RFMOs,\\
%       \h{25.3ex}SPC staff positions \& consultants}\\[1ex]
%   \end{itemize}
% \end{frame}
% 
% % ______________________________________________________________________________
% 
% \begin{frame}{SC20 discussion}
%   \begin{itemize}
%     \item Whether SPC should migrate upcoming {\darkgreen\bf billfish
%       assessments} to Stock Synthesis\\
%     \comment{\gray swordfish 2025, striped marlin 2029}\\[4ex]
%     \item Select {\darkgreen\bf scoping project} tasks to prioritize in
%     2024--2026\\
%     \comment{\gray from the list of 10 tasks in the report, or other
%       tasks}\\[4ex]
%     \item What is needed to launch the {\darkgreen\bf main project} and when\\
%     \comment{\gray conducting model exploration and software development, TORs,
%       resources}\\[3ex]
%   \end{itemize}
% \end{frame}
% 
% % ______________________________________________________________________________
% 
% \begin{frame}{Stock Synthesis}
%   \textit{Cons}
%   \begin{itemize}
%     \item[--] expected to be phased out in the not-too-distant future\\[1ex]
%     \item[--] fewer features than MFCL\\[3ex]
%   \end{itemize}
%   \textit{Pros}
%   \begin{itemize}
%     \item[+] used by IATTC, IOTC, ICCAT, and ISC (and NOAA, ICES, GFCM,
%     etc.)\\[1ex]
%     \item[+] facilitates collaboration between the tuna RFMOs, including future
%     development\\[1ex]
%     \item[+] shortens training time for new SPC staff, makes skills and
%     experience transferable\\[1ex]
%     \item[+] large user community, relevant for peer reviews and discussing
%     technical decisions\\[1ex]
%     \item[+] exceptionally complete suite of tools, diagnostics, automated plots
%     and tables\\[1ex]
%     \item[+] next-generation frameworks will support transitioning from Stock
%     Synthesis\\[1ex]
%   \end{itemize}
% \end{frame}
% 
% % ______________________________________________________________________________
% 
% \begin{frame}{Recommendations from 2024 international expert meeting}
%   \begin{enumerate}
%     \item {\darkgreen\bf Tuna} assessment software\\
%     \comment{\gray design and develop a model specific for tuna
%       assessments}\\[1ex]
%     \item {\darkgreen\bf RTMB} programming environment\\
%     \comment{\gray lean software development paradigm, maybe a specific model
%       for each species}\\[1ex]
%     \item {\darkgreen\bf State-space} formulation\\
%     \comment{\gray statistically and computationally efficient way to allow
%       time-varying processes}\\[1ex]
%     \item {\darkgreen\bf Age-length} structure\\
%     \comment{\gray explicitly track the population by age and length, if not too
%       costly}\\[1ex]
%     \item {\darkgreen\bf Simple} models\\
%     \comment{\gray short-term staff, young scientists, simple user interface,
%       simpler models}\\[1ex]
%     \item {\darkgreen\bf Collaboration} between tuna RFMOs\\
%     \comment{\gray MFCL and Stock Synthesis in a sunset phase, data analyses
%       comparable between RFMOs}\\[1.5ex]
%   \end{enumerate}
% \end{frame}
% 
% % ______________________________________________________________________________
% 
% \begin{frame}{Stock assessment software}
%   Existing software, ready for multi-region tuna assessments\\[3ex]
%   \begin{itemize}
%     \item[-] {\darkgreen\bf Stock Synthesis} is used by IATTC, IOTC, and
%     ICCAT\\[3ex]
%     \item[-] {\darkgreen\bf Gadget} has many features relevant for tuna
%     assessments\\[3ex]
%     \item[-] {\darkgreen\bf Casal} has many features relevant for tuna
%     assessments\\[4ex]
%   \end{itemize}
%   These could be extended further as needs arise\\[2ex]
% \end{frame}
% 
% % ______________________________________________________________________________
% 
% \begin{frame}{Stock assessment software}
%   Software that could be developed further:\\[3ex]
%   \begin{itemize}
%     \item[-] {\darkgreen\bf sbt} is built around CKMR, currently for
%     single-region assessments\\[3ex]
%     \item[-] {\darkgreen\bf ALSCL} is a state-space model that fits length
%     comps, currently no catches\\[3ex]
%     \item[-] {\darkgreen\bf WHAM$\,$\raisebox{0.15ex}{+}$\,$Length} is a
%     state-space that fits length comps, currently single-region\\[3ex]
%     \item[-] {\darkgreen\bf SAM$\,$\raisebox{0.15ex}{+}$\,$Length} is an early
%     exploration of extending SAM to fit length comps\\[3ex]
%     \item[-] {\darkgreen\bf Stock Synthesis$\,$\raisebox{0.15ex}{+}$\,$Enhanced
%       Tags} is a proposed enhancement of the tag module\\[2ex]
%   \end{itemize}
% \end{frame}
% 
% % ______________________________________________________________________________
% 
% \begin{frame}{Stock assessment software}
%   Also relevant:\\[4ex]
%   \begin{itemize}
%     \item[-] {\darkgreen\bf Stock Synthesis$\,$\raisebox{0.15ex}{+}$\,$CKMR} is
%     an experimental add-on, not included in core software\\[4ex]
%     \item[-] {\darkgreen\bf FIMS}, NOAA project coordinating the development of
%     a next-generation framework\\[6ex]
%   \end{itemize}
% \end{frame}
% 
% % ______________________________________________________________________________
% 
% \begin{frame}{Possible tasks for SPC to prioritize}
%   Subject to SC advice and funding approvals by WCPFC:\\[4ex]
%   {\orange\it Migrate assessments to existing software}\\[2ex]
%   \begin{enumerate}
%     \item Move the {\darkgreen swordfish} assessment to Stock Synthesis\\
%     \comment{\gray relatively simple compared to other SPC assessments}\\[3ex]
%     \item Move the {\darkgreen striped marlin} assessment to Stock Synthesis\\
%     \comment{\gray also relatively simple}\\[3ex]
%   \end{enumerate}
%   \gray stepwise: previous MFCL diagnostic $\Rightarrow$ catch-conditioned
%   MFCL $\Rightarrow$ Stock Synthesis
% \end{frame}
% 
% % ______________________________________________________________________________
% 
% \begin{frame}{Possible tasks for SPC to prioritize}
%   Subject to SC advice and funding approvals by WCPFC:\\[3.5ex]
%   {\orange\it Model exploration using existing software}\\[2ex]
%   \begin{enumerate}\setcounter{enumi}{2}
%     \item Explore Casal/Gadget/Stock Synthesis/sbt models for {\darkgreen
%       albacore}\\
%     \comment{\gray simpler than the other tuna species}\\[3ex]
%     \item Explore Casal/Gadget/Stock Synthesis models for original {\darkgreen
%       five-region yellowfin} data\\
%     \comment{\gray test capabilities of platforms: regions, tags, large number
%       of fisheries}\\[3ex]
%     \item Explore a variety of models for a simplified {\darkgreen single-region
%       yellowfin} tuna dataset\\
%     \comment{\gray ALSCL, Casal, Gadget, MFCL, sbt, Stock Synthesis,
%       WHAM\raisebox{0.15ex}{+}Length}\\[3ex]
%   \end{enumerate}
% \end{frame}
% 
% % ______________________________________________________________________________
% 
% \begin{frame}{Possible tasks for SPC to prioritize}
%   Subject to SC advice and funding approvals by WCPFC:\\[3ex]
%   {\orange\it Extend existing software}\\[1.5ex]
%   \begin{enumerate}\setcounter{enumi}{5}
%     \item ALSCL$\,$\raisebox{0.15ex}{+}$\,$Fleets\\
%     \comment{\gray Fan Zhang (Shanghai Ocean University) and Nick Davies (SPC
%       consultant)}\\[1.5ex]
%     \item Stock Synthesis$\,$\raisebox{0.15ex}{+}$\,$Enhanced Tags\\
%     \comment{\gray Nicholas Ducharme-Barth, Matthew Vincent (NOAA), and Arni
%       Magnusson (SPC)}\\[1.5ex]
%     \item WHAM$\,$\raisebox{0.15ex}{+}$\,$Length\\
%     \comment{\gray Giancarlo Correa (AZTI) and Arni Magnusson (SPC)}\\[1.5ex]
%     \item SAM$\,$\raisebox{0.15ex}{+}$\,$Length\\
%     \comment{\gray Anders Nielsen (DTU), Colin Millar (ICES), and Arni Magnusson
%       (SPC)}\\[1.5ex]
%   \end{enumerate}
% \end{frame}
% 
% % ______________________________________________________________________________
% 
% \begin{frame}{Possible tasks for SPC to prioritize}
%   Subject to SC advice and funding approvals by WCPFC:\\[4ex]
%   {\orange\it Design and develop new software for tuna assessments}\\[2ex]
%   \begin{enumerate}\setcounter{enumi}{9}
%     \item Initial explorations using {\darkgreen RTMB}\\
%     \comment{\gray Nick Davies (SPC consultant) and Arni Magnusson (SPC)}
%   \end{enumerate}
% \end{frame}
% 
% % ______________________________________________________________________________
% 
% \begin{frame}{Possible tasks for SPC to prioritize}
%   \begin{enumerate}
%     \item Move the {\darkgreen swordfish} assessment to Stock Synthesis\\[1ex]
%     \item Move the {\darkgreen striped marlin} assessment to Stock
%     Synthesis\\[1ex]
%     \item Explore Casal/Gadget/Stock Synthesis/sbt models for {\darkgreen
%       albacore}\\[1ex]
%     \item Explore Casal/Gadget/Stock Synthesis models for original {\darkgreen
%       five-region yellowfin} data\\[1ex]
%     \item Explore a variety of models for a simplified {\darkgreen single-region
%       yellowfin} tuna dataset\\[1ex]
%     \item ALSCL$\,$\raisebox{0.15ex}{+}$\,$Fleets\\[1ex]
%     \item Stock Synthesis$\,$\raisebox{0.15ex}{+}$\,$Enhanced Tags\\[1ex]
%     \item WHAM$\,$\raisebox{0.15ex}{+}$\,$Length\\[1ex]
%     \item SAM$\,$\raisebox{0.15ex}{+}$\,$Length\\[1ex]
%     \item Initial explorations using {\darkgreen RTMB}\\[1ex]
%   \end{enumerate}
% \end{frame}
% 
% % ______________________________________________________________________________
% 
% \begin{frame}{Project website}
%   \centering\small
%   \textblue{\url{https://github.com/PacificCommunity/ofp-sam-transition-plan}}
% \end{frame}

\end{document}
